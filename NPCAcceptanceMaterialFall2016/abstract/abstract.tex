%\documentclass[11pt]{article}
\documentclass[12pt,letterpaper,oneside,openright]{book}
\usepackage[letterpaper]{geometry}
\usepackage[none]{hyphenat}
\usepackage{libertine}

\begin{document}
\chapter*{\centering Abstract}
\begin{sloppypar}
Many modern long baseline neutrino oscillation experiments use the $\nu_\mu$/$\bar{\nu}_\mu$-nucleus charged-current (CC) interactions to infer oscillation parameters. The NOvA experiment uses the off-axis NuMI beam with a neutrino energy spectrum peaking at about 2 GeV. In this energy range, the quasielastic, the resonant production, and the deep inelastic scattering processes come into play in the total cross section. An inclusive measurement including all channels, together with measurements of individual channels, could provide a more complete picture. I will be analysing a small dataset of the reversed horn current (RHC) data taken right before the 2016 summer shutdown. This dataset could not only shed light on questions remaining for the forward horn current (FHC) runs, but also provide a measurement of the $\bar{\nu}_\mu$-nucleus CC inclusive cross section measurement, which is even less measured than that for $\nu_\mu$ in NOvA's energy range.
\end{sloppypar}
\end{document}