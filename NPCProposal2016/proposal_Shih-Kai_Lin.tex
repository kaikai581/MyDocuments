%%%%%%%%%%%%%%%%%%%%%%%%%%%%%%%%%%%%%%%%%
% Thin Sectioned Essay
% LaTeX Template
% Version 1.0 (3/8/13)
%
% This template has been downloaded from:
% http://www.LaTeXTemplates.com
%
% Original Author:
% Nicolas Diaz (nsdiaz@uc.cl) with extensive modifications by:
% Vel (vel@latextemplates.com)
%
% License:
% CC BY-NC-SA 3.0 (http://creativecommons.org/licenses/by-nc-sa/3.0/)
%
%%%%%%%%%%%%%%%%%%%%%%%%%%%%%%%%%%%%%%%%%

%----------------------------------------------------------------------------------------
%	PACKAGES AND OTHER DOCUMENT CONFIGURATIONS
%----------------------------------------------------------------------------------------

\documentclass[a4paper, 11pt]{article} % Font size (can be 10pt, 11pt or 12pt) and paper size (remove a4paper for US letter paper)

\usepackage[letterpaper]{geometry}
%\usepackage[protrusion=true,expansion=true]{microtype} % Better typography
\usepackage{graphicx} % Required for including pictures
\usepackage{wrapfig} % Allows in-line images
\usepackage{hyperref} % Enable URL

%\usepackage{mathpazo} % Use the Palatino font
\usepackage[T1]{fontenc} % Required for accented characters
\usepackage{libertine}
\usepackage{multirow} % http://ctan.org/pkg/multirow
\usepackage{hhline} % http://ctan.org/pkg/hhline
%\linespread{1.05} % Change line spacing here, Palatino benefits from a slight increase by default

\makeatletter
\renewcommand\@biblabel[1]{\textbf{#1.}} % Change the square brackets for each bibliography item from '[1]' to '1.'
\renewcommand{\@listI}{\itemsep=0pt} % Reduce the space between items in the itemize and enumerate environments and the bibliography

\renewcommand{\maketitle}{ % Customize the title - do not edit title and author name here, see the TITLE block below
\begin{flushright} % Right align
{\LARGE\@title} % Increase the font size of the title

\vspace{50pt} % Some vertical space between the title and author name

{\large\@author} % Author name
\\\@date % Date

\vspace{40pt} % Some vertical space between the author block and abstract
\end{flushright}
}

%----------------------------------------------------------------------------------------
%	TITLE
%----------------------------------------------------------------------------------------

\title{\textbf{Research Proposal for the NPC Fellowship}\\ % Title
} % Subtitle

\author{\textsc{Shih-Kai Lin} % Author
\\{\textit{Postdoc at Colorado State University}} % Institution
\\{\textsc{Principal Investigator: Prof. Norm Buchanan}}}

\date{} % Date

%----------------------------------------------------------------------------------------

\begin{document}

\maketitle % Print the title section

%----------------------------------------------------------------------------------------
%	ABSTRACT AND KEYWORDS
%----------------------------------------------------------------------------------------

%\renewcommand{\abstractname}{Summary} % Uncomment to change the name of the abstract to something else

%\begin{abstract}
%Morbi tempor congue porta. Proin semper, leo vitae faucibus dictum, metus mauris lacinia lorem, ac congue leo felis eu turpis. Sed nec nunc pellentesque, gravida eros at, porttitor ipsum. Praesent consequat urna a lacus lobortis ultrices eget ac metus. In tempus hendrerit rhoncus. Mauris dignissim turpis id sollicitudin lacinia. Praesent libero tellus, fringilla nec ullamcorper at, ultrices id nulla. Phasellus placerat a tellus a malesuada.
%\end{abstract}
%
%\hspace*{3,6mm}\textit{Keywords:} lorem , ipsum , dolor , sit amet , lectus % Keywords
%
%\vspace{30pt} % Some vertical space between the abstract and first section

%----------------------------------------------------------------------------------------
%	ESSAY BODY
%----------------------------------------------------------------------------------------
\paragraph{\textbf{Title}}\hspace*{\fill}\\
Study of muon antineutrino charged-current interactions and management of data-driven trigger for NOvA

\paragraph{}\hspace*{\fill}\\
The proposed research has two goals. The first goal is to study the muon antineutrino interactions with the NOvA near detector data taken with horn current reversed before the 2016 summer shutdown of the neutrino beam.

A short period of NOvA reversed horn current (RHC) data was taken from June 30, 2016 to July 30, 2016. The purpose of this dataset is to give information for NOvA's Monte Carlo tuning and physics modeling~\cite{Patterson:2016_1}. NOvA's neutrino beam energy, which is tuned for long baseline oscillation measurements, sits on an intricate energy range where all the major neutrino-nucleus interaction modes, namely quasielastic, resonant production, and deep inelastic scattering, have a significant contribution to the total interaction cross-section. In analyzing the $\nu_\mu$ data, the collaboration found that there were significant inconsistencies between simulation and data in hadronic energy estimation, and these inconsistencies led to large systematic uncertainties in NOvA's first oscillation analyses~\cite{NOvA:2016_1}~\cite{NOvA:2016_2}.

There have been significant improvements in hadronic energy estimation since NOvA's first analyses, including the incorporation of the meson exchange current (MEC) model into the GENIE neutrino generator, a reduction in the rate of nonresonant single pion production, and a reassignment to the uncertainties for charged-current (CC) quasielastic scattering~\cite{Wolcott:2016_1}. However, even with these major improvements, discrepancies remain between data and simulation. There is still a $2.5\%$ shift in the reconstructed neutrino energy, and there are still data/MC discrepancies particularly in  high-$y$, low muon track length situations. Antineutrino data, with different interaction cross section and $y$ distribution, could shed lights on these issues.

In addition to improvements to the oscillation analysis, investigations also show that it is feasible to perform a CC inclusive cross section measurement. Furthermore, statistics were large enough during the RHC that a double differential measurement may be possible. Since most modern long baseline neutrino oscillation experiments use $\nu_\mu/\bar{\nu}_\mu$ beam, the inclusive cross section measurement could not only contribute to NOvA, but such a measurement would be of interest to the neutrino community in general. In addition, this measurement could serve as one of the first measurements in the energy region of NOvA on the global plot~\cite{Zeller:2015_1}. This measurement includes studies on event selection and its efficiency, neutrino flux, number of target nucleons, energy scale, background, unfolding, GENIE physics modeling, and corresponding uncertainties. Some of the items share results with the $\nu_\mu$ CC inclusive measurement, while others require independent studies.

GENIE physics modeling is particularly interesting among the items. There is evidence of 2p2h effects also in the RHC data~\cite{Bashar:2016_1}. A data/MC comparison shows that the inclusion of MEC effect leads to better data MC agreement for the forward horn current (FHC) data. However, the current MEC model for $\bar{\nu}_\mu$ only partially improves the data/MC agreement - disagreement is greater at low visible hadronic energy. One hypothesis is that the NOvA detectors might see neutrons, and methods for measuring and reconstructing neutrons are under investigation. Our group at Colorado State University plans to take on this problem by looking at the event topology at the very low visible hadronic energy and see if any event topology in data but not in MC can be identified. MEC effects in other variables such as the invariant hadronic mass $W$ and the four-momentum transfer $Q$ will also be pursued.

The second goal of the proposed research is to provide improvements to the data-driven trigger (DDT) - specifically involving code and package management. One of the innovative features of NOvA's data acquisition (DAQ) is that the detectors operate in a trigger-less mode, continuously sending data to the buffer nodes with capacity of at least $\sim 20$ s worth of data. The system then waits for the information of the beam spill to arrive and makes trigger decisions. Keeping tens of seconds of data in the buffer node farm opens up the capability of searching for interesting events not associated with the beam pulses. This kind of trigger is NOvA's data-driven trigger.

The DDT code base has been managed in a per package manner, meaning that the individual author of the packages takes the responsibility that his or her code builds and takes sensible data. This model of DDT package management has become cumbersome and prevents careful validation from routinely occurring. One of the priorities of the DDT group is to reorganize the DDT software packages to centralize common algorithms like clustering and tracking. Such improvement will make the code easier to manage and make it easier for new group members to quickly become engaged in trigger development. The DDT and DAQ groups also plan to migrate to the SRT build system from the multi release build (MRB) system. Using SRT could avoid inconsistencies between the online and the offline DDT builds. I am taking the role as the manager of the DDT code base and working on the build system migration.
%However, past experience shows this way sometimes leads to code redundancy, such as a same reconstruction algorithm being implemented in different packages. Besides, the lack of validation procedures also resulted in nonsensical data. The DDT group has come up with the idea of cleaning up the DDT codebase, making it more modular, and having a person to watch over the consistency between packages and tag releases. Also, the developers are requested to follow a validation procedure before deployment, and a person will be supervising this process. I am taking this role as the code manager. Furthermore, since the current DDT uses multi release build (MRB) as its build system, which leads to inconsistency between the online and the offline DDT builds, the DAQ and the DDT group plan to migrate the build system back to the older SRT. I will also be working on this together with people from DAQ and DDT group.\newline
I will be working closely with one of the conveners of the NOvA near detector physics group, Jonathan Paley, on $\bar{\nu}_\mu$ interactions, and with Andrew Norman on DDT management.\newline

Office space with a computer for code development and software for data analysis are readily accessible. To cover the move from my home institution in Fort Collins to Chicago Area, which costs about \$2,400, and local expenses, which cost about \$1,600 monthly including housing and utilities, I am requesting a budget of \$\underline{12,000} for a six-month period.
%\$\underline{\hspace{2cm}}.
\clearpage

\paragraph{\textbf{Proposed Timeline}}\hspace*{\fill}\\

\begin{table}[!htbp]
  \centering
  \begin{tabular}{|c|c|c|c|}
    \hline
    time & $\bar{\nu}_\mu$ interaction \& cross section & \multicolumn{2}{c|}{data-driven trigger} \\
    \hline
    2 weeks & event selection \& efficiency & \multirow{5}{*}{code management} & \multirow{3}{*}{SRT transition} \\
    \hhline{--~~}
    2 weeks & neutrino flux, target nucleons & & \\
    \hhline{--~~}
    4 weeks & systematic uncertainties & & \\
    \hhline{--~-}
    4 weeks & background & & \\
    \hhline{--~}
    12 weeks & MEC for RHC, energy scale & & \\
	\hline
  \end{tabular}
  %\caption{R, C ripple size}
  \label{T:peak}
\end{table}

%----------------------------------------------------------------------------------------
%	BIBLIOGRAPHY
%----------------------------------------------------------------------------------------

\bibliographystyle{unsrt}

\bibliography{sample}

%----------------------------------------------------------------------------------------

\end{document}