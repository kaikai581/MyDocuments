\chapter{Introduction}
The goal of this dissertation is to study neutron production in liquid scintillator by cosmic ray muons with the Daya Bay reactor neutrino experiment.

Underground experiments searching for rare events are sensitive to background. Muons are highly penetrating particles which not only can reach underground laboratories, they could also interact with surrounding material of the experiment, and produce secondary particles or isotopes which contribute to experimental background. Among these secondary particles, neutrons are one of the most important.

Daya Bay is a neutrino oscillation experiment aimed at measuring the parameter $\theta_{13}$ in the neutrino mixing matrix to high precision. Daya Bay utilizes the inverse beta decay (IBD) reaction, involving a neutrino interacting with a proton, producing a positron and a neutron. In Daya Bay, cosmogenic neutrons can cause accidental events, mimicing the inverse beta decay by first scattering with a proton and later getting captured. Neutron background in Daya Bay is suppressed by locating the detectors underground, and surrounding the detectors with a minimum of 2.5 meters of water. Daya Bay's muon system consists of a water Cherenkov system and Resistive Plate Chambers (RPCs).

Neutron background can be measured in-situ in Daya Bay, and can be estimated by several computer programs. In order to reach the ultimate sensitivity when measuring $\theta_{13}$, neutron background has to be known to high precision. The Daya Bay neutrino experiment is a very good detector to study neutron production by muons. Daya Bay has a good muon system with very high muon tagging efficiency. Muons are tracked by the muon detectors and the antineutrino detectors. Since Daya Bay detects the neutrino through the time coincidence of positron and neutron signals, the main detectors are required to act as neutron detectors also. This requirement makes Daya Bay's main detectors very good neutron detectors. The neutron capture vertices can be reconstructed accurately. Because the main detectors register the energy deposited by muons, mostly ionization energy by muons which don't interact with nuclei. For those having nuclear interactions, Daya Bay can measure the energy deposited by secondary particles. 

In this work, we present a method to measure the neutron yield utilizing the advantages Daya Bay offers. The conventional way is based heavily on Monte Carlo simulations. However, the neutron-nucleus interaction process is complex to simulate and model dependent. The method we used here will make a well-defined cylindrical fiducial volume of fixed radius around each muon track. The length of the cylinder is such that the cylinder is fully contained inside the neutrino detector. In this way, the loss of neutrons in the back of the cylinder is partially compensated by those gained in the front. On the other hand, the radial loss can be estimated by measuring the radial dependence. We believe this track-by-track method frees us from the dependence on Monte Carlo simulations. In principle, it can supply a validation of the Monte Carlo. 

In Chapter 2 we give a brief introduction to neutrino physics and what Daya Bay aims to measure. We describe the Daya Bay reactor neutrino experiment in Chapter 3. In Chapter 4 we describe the Daya Bay data acquisition. The Daya Bay RPC is described in Chapter 5. In Chapter 6 we describe mechanisms of neutron production in muon-nucleus interactions. The analysis and results are presented in Chapter 7. The summary and conclusion are given in Chapter 8.