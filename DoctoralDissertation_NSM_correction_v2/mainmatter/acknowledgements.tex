\chapter*{\centering Acknowledgements}
\addcontentsline{toc}{chapter}{Acknowledgements}
I would like to express my deepest appreciation to my thesis adviser, Dr. Kwong Lau, who during these years demonstrated how to think and act as a physicist. I learned to develop a sense of numbers and to do orders of magnitude estimation before any careful calculation. This way I gain deeper physics insights without getting lost in complicated equations and numerous numbers. Moreover, he taught me the attitudes a physicist should hold. First, always doubt. It is not only because in the history of science many breakthroughs were brought about by careful inspection of previous works, but because only after I manage to verify the previous works do I gain deeper understanding. Second, be bold. Do not hesitate to show something new even if it is not fully understood. After all, there is no perfect analysis; one can always improve it. In addition, I have to thank him for sending me to Daya Bay onsite, where I participated in many Daya Bay installation activities and gained precious experience in the construction of a large physics project.

I would like to thank my committee members for the insightful opinions on my work. Dr. Bellwied's advanced particle physics class sharpens and broadens my knowledge in particle physics. Dr. Hungerford's valuable inputs on my neutron study introduced me to the Monte Carlo studies on this difficult topic. Dr. Pinsky, who is also my Daya Bay colleague, had enlightening conversation with me about physics in general.

I have to express thanks to my Daya Bay colleagues, Dr. Kam-Biu Luk and Dr. Randy Johnson, with whom I worked at Daya Bay and gained valuable hardware experience in PMTs. I also would like to thank Dr. Lisa Whitehead, who have convened the meeting of the cosmogenic background working group, where I had the chance to discuss with experts about my analysis.

Finally, I have to thank my wife, Wen-Wen Luan, for her regretless support of our family. I thank my kid, Kevin Lin, for joining our family in the second year of my Ph.D. study to enrich our lives. Without them, I would not be able to persist and earn the degree.

This work was supported by the US Department of Energy.
\newpage