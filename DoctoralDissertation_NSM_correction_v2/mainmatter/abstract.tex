\chapter*{\centering Abstract}
The Daya Bay Reactor Neutrino Experiment is designed to measure the neutrino mixing angle, $\theta_{13}$, with a sensitivity of $\sin^22\theta_{13}<0.01$ at $90\%$ confidence level. Neutrons produced by cosmic muon spallation constitute one of the main backgrounds, and an understanding of the neutron yield of a muon is important. Since Daya Bay has a very good muon tracking system, muon tracks can be reconstructed accurately. The neutron yield can be determined by constructing a fiducial volume around each muon track. The neutron yield is found to be of the order of $10^{-4}cm^2/\mu/g$ for each experimental hall of Daya Bay. The results show a gradual dependence on muon energy, and are in agreement with other measurements. Since Daya Bay is capable of measuring energy deposition, the neutron events show enhanced energy deposition.
\addcontentsline{toc}{chapter}{Abstract}