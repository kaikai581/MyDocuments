%\documentclass[11pt]{article}
\documentclass[12pt,letterpaper,oneside,openright]{book}
\usepackage[letterpaper]{geometry}
\usepackage[none]{hyphenat}
\usepackage{libertine}

\begin{document}
\chapter*{\centering Abstract}
\begin{sloppypar}
The NOvA (NuMI Off-axis $\nu_e$ Appearance) experiment is a long-baseline neutrino oscillation experiment, looking primarily for $\nu_\mu\rightarrow \nu_e$ appearance and $\nu_\mu\rightarrow \nu_\mu$ disappearance. NOvA consists of two functionally identical detectors placed 14 mrad off-axis of the NuMI (Neutrinos at the Main Injector) beam line, with a near detector underground at Fermilab for constraining the flux and a far detector 810 km away at Ash River, MN. The $\nu_e$ appearance measurement at NOvA aims at resolving neutrino mass hierarchy and constraining the CP phase, while the $\nu_\mu$ disappearance measurement is sensitive to the octant of the mixing angle $\theta_{23}$. This talk will discuss NOvA's results which will lead the way to the next generation neutrino experiment, DUNE, at SURF.
\end{sloppypar}
\end{document}